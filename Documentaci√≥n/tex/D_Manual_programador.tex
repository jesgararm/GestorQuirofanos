\apendice{Documentación técnica de programación}

\section{Introducción}

Para realizar un correcto \textbf{análisis} del desarrollo de este proyecto, son necesarias algunas consideraciones respecto al entorno de desarrollo, las dependencias funcionales y las opciones de integración y despliegue.

\section{Estructura de directorios}

Dentro del repositorio, podemos encontrar:
\begin{itemize}
    \item \textit{./}: Directorio raíz. Contiene la licencia, el logo, el fichero \textit{léeme} para la página principal de GitHub, la estructura de base datos para su importación y todos los paquetes y directorios que conforman el proyecto.
    \item \textit{./API}:  Sistema que contiene la API con los servicios de Predicción y Planificación quirúrgica.
    \item ./\textit{API}/\textit{uploads}: Directorio de almacenamiento temporal que recoge los archivos enviados por los usuarios antes de ser procesados y eliminados.
    \item \textit{./API/src}: Código fuente de la API.
    \item \textit{./API/src/scheduling}: Contiene las clases y paquetes encargados de la tarea de planificación, así como el archivo Dockerfile para ser encapsulado.
    \item \textit{./API/src/scheduling/Optimizacion}: Directorio con clases y paquetes con los algoritmos de optimización y planificación.
    \item \textit{./API/src/scheduling/Optimizacion/Genético}: Clases para el algoritmo genético.
    \item \textit{./API/src/scheduling/Optimizacion/Heurísticas}: Clases para las heurísticas de planificación.
    \item \textit{./API/src/predictions}: Directorio con clases encargadas de cargar el modelo de ML predictivo.
    \item \textit{./API/src/common}: Clases comunes a todos los paquetes. Encargados fundamentalmente de tareas de procesamiento de datos y ficheros.
    \item \textit{./APP-WEB}: Contiene el Dockerfile y el código fuente de la interfaz web.
    \item \textit{./APP-WEB/src}: Código fuente de la aplicación.
    \item \textit{./APP-WEB/src/admin}: Contiene las rutas a las funcionalidades del usuario administrador.
    \item \textit{./APP-WEB/src/auth}: Contiene las rutas para la funcionalidad de autenticación en sistema.
    \item \textit{./APP-WEB/src/forms}: Contiene los formularios en formato Flask WTF para incluirlos en las plantillas HTML.
    \item \textit{./APP-WEB/src/models}: Clases y paquetes que encapsulan el modelo de comunicación con la base de datos.
    \item \textit{./APP-WEB/src/models/entities}: Contiene las clases que representan las entidades persistentes de la BBDD.
    \item \textit{./APP-WEB/src/public}: Contiene las rutas para todas las funcionalidades disponibles para los usuarios.
    \item \textit{./APP-WEB/src/static}: Contiene los archivos de formato de estilos, imágenes... a incluir en las plantillas estáticas.
    \item \textit{./APP-WEB/src/templates}: Contiene las plantillas en lenguaje de marcado HTML, que se corresponden con la \textit{Vista} del usuario.
    \item \textit{./APP-WEB/src/templates/admin}: Plantillas dedicadas a la vista del administrador.
    \item \textit{./APP-WEB/src/templates/user}: Plantillas para la vista de cualquier usuario.
    \item \textit{./APP-WEB/src/templates/auth}: Plantillas para la función de identificación en el sistema.
    \item \textit{./Documentación}: Se incluye la memoria, anexos y diagramas.
    \item \textit{./Documentación/img}: Ruta de las imágenes que se encuentran en el entregable.
    \item \textit{./Documentación/tex}: Capítulos y apartados de memoria y anexos, en formato latex.
    \item \textit{./Documentación/UML}: Directorio con diagramas de casos de uso, interacción y secuencia.
    \item \textit{./Experimentación}: Colección de Jupyter Notebooks que ilustran el proceso de investigación llevado a cabo hasta obtener la solución propuesta.
    \item \textit{./Experimentación/Datos}: Colección de conjuntos de datos anonimizados para labores de ML.
    \item \textit{./Experimentación/Modelos}: Análisis, diseño y explotación de modelos predictivos de aprendizaje supervisado.
    \item \textit{./Experimentación/Optimización}: Análisis, diseño y explotación de algoritmos de planificación paralela.
    \item \textit{./Experimentación/Preprocesado:} Herramientas de preprocesamiento de datos a partir de los listados originales.
    \item \textit{./Preprocesado}: Clase con utilidades para estandarizar y homogeneizar las fuentes de datos en un formato compatible con los modelos propuestos.
\end{itemize}


\section{Manual del programador}

En este apartado desglosaremos la estructura del sistema de forma que sirva de referencia para futuros desarrolladores para su análisis y contribución al proyecto.

Para ello, haremos referencia al \textbf{entorno y dependencias} necesarias para el desarrollo, la obtención del \textbf{código fuente}, su \textbf{ejecución} y posterior \textbf{exportación}.

\subsection{Entorno de desarrollo}

Para comenzar con la \textit{explotación} del sistema, se recomiendan las siguientes herramientas y dependencias:

\subsubsection{Python 3.8-3.11}

En nuestro caso hemos usado la última versión ofrecida por \textbf{Anaconda}, pues ya incluye gran parte de las librerías necesarias para las tareas de análisis y minería de datos, cuya guía de instalación se encuentra disponible \href{https://docs.anaconda.com/free/anaconda/install/index.html}{aquí}.

\subsubsection{Virtualenv}

Incluido en la mayor parte de distribuciones de Python (\textit{incluido en Anaconda}). Permite trabajar con entornos \textit{virtuales}, de gran utilidad cuando construimos subsistemas exportables y queremos \textbf{acotar} las librerías necesarias para cada entorno.

En caso de no estar disponible, puede instalarse con el comando pip:
\texttt{\textit{pip -U install virtualenv}}

\subsubsection{IDEs}

Se recomienda la instalación de una interfaz de apoyo al desarrollo compatible con Python.
Recomendamos la instalación de \href{https://code.visualstudio.com/download}{Visual Studio Code}, la \href{https://marketplace.visualstudio.com/items?itemName=ms-python.python}{extensión de Python} y el \href{https://marketplace.visualstudio.com/items?itemName=HansUXdev.bootstrap5-snippets}{plugin de plantillas Bootstrap 5}.

Por otra parte, el IDE Pycharm, en su versión \textbf{Professional}, incluye soporte y apoyo al desarrollo web con Flask, añadiendo integración con HTML, JS y SQL y está disponible de forma \textit{gratuita} para la comunidad educativa y es accesible desde \href{https://www.jetbrains.com/es-es/pycharm/download/#section=windows}{aquí}.

Sin embargo, pese a haber explorado ambos IDE para la confección del código fuente, la totalidad del proyecto puede ser desarrollado desde cualquiera de ellos, sin necesidad de combinar su uso.

\subsubsection{Git}

Necesario para hacer uso del \href{https://github.com/jesgararm/GestorQuirofanos}{repositorio}. Nuestro repositorio es público, recomendándose realizar un \textit{fork} del mismo en una cuenta privada de \textit{GitHub} y trabajar desde una copia privada del mismo en nuestro entorno:

\imagen{fork}{Vista del repositorio y función de fork}

Por otro lado, es recomendable disponer de \textit{git} instalado en el computador principal, para poder acceder al código fuente y poder realizar modificaciones con cambios persistentes.

Recomendamos para tal índole la instalación de la suite \href{https://docs.github.com/en/desktop/installing-and-configuring-github-desktop/installing-and-authenticating-to-github-desktop/installing-github-desktop}{Github Desktop}, pues permite desde un interfaz sencillo y comprensible realizar la clonación del repositorio en un equipo local y manejar las modificaciones y las \textit{ramas} de trabajo sin necesidad de conocer los parámetros y funcionalidades \textit{git} desde la línea de comandos.

\subsubsection{Sistema Gestor de Bases de Datos (SGBD)}

Es recomendable, para realizar pruebas y comprender el diseño de datos, contar con un sistema gestor de base de datos \textbf{compatible con MySQL e InnoDB}.

Recomendamos la instalación de \href{https://www.apachefriends.org/es/index.html}{XAMPP}, que es un entorno de desarrollo PHP que incluye el SGBD phpmyadmin. Desde ese entorno hemos diseñado y realizado las pruebas locales del sistema, previa a su exportación del esquema en fichero sql.

\subsubsection{Docker}

Dado que usaremos contenedores para encapsular la lógica del sistema y, posteriormente, desplegarlos para su funcionamiento desde cualquier computador, debemos tener el \textit{daemon} Docker instalado en nuestro sistema.

Al igual que con \textit{git}, recomendamos la instalación del GUI oficial de Docker, \href{https://docs.docker.com/desktop/install/windows-install/}{Docker Desktop}. 


\subsection{Obtención del código fuente}

Una vez instalado \textit{\textbf{Github Desktop}} y realizado un \textit{fork} del repositorio, podemos obtener su contenido desde el propio interfaz: \texttt{File > Clone Repository}

\imagen{githubdesktop}{Clonar Repositorio con Github Desktop}

Por otro lado, desde \textit{Git Bash}, abriendo la terminal en el directorio local donde deseemos obtener la copia del código fuente, bastará con ejecutar el siguiente comando:

\texttt{git clone https://github.com/jesgararm/GestorQuirofanos.git}

\subsection{Importación del proyecto e instalación de dependencias}

Dado que Python es un lenguaje \textit{interpretado}, no es necesaria su compilación, por lo que los IDEs no requieren de proyectos con dependencias funcionales preestablecidas (\textit{al contrario que otros desarrollados en lenguajes compilados, como Java}).

Para empezar a trabajar en el repositorio clonado, basta con añadir al entorno de trabajo de nuestro IDE la localización local del repositorio. Por ejemplo, desde VS Code:
\texttt{File > Add Folder to Workspace}

\imagen{vscodefolderpng}{Agregar repositorio local al entorno de trabajo de Visual Studio Code}

\subsubsection{Entornos virtuales e instalación de librerías y dependencias}

Python y la librería \textit{virtualenv} nos permiten trabajar con \textbf{entornos virtuales}. Esto nos permite instalar y trabajar con entornos de python \textbf{aislados}, cada uno con sus librerías y dependencias independientes del paquete que tengamos instalados en la raíz de nuestro sistema.

Cabe destacar que, una vez instalado y activado un entorno, todas las dependencias agregadas al mismo serán \textit{exclusivas} de éste y no serán duplicadas en el entorno principal. Del mismo modo, las librerías principales no serán accesibles desde el nuevo entorno, debiendo configurarlo y definirlo desde 0 tras su activación.

Recomendamos la creación de \textbf{dos entornos virtuales}, uno para cada subsistema. Para ello, nos situaremos en el directorio \textit{./API} o \textit{./APP-WEB} y ejecutaremos el comando:

\texttt{virtualenv [nombreentorno]}

Una vez realizado, pasaremos a su activación, ejecutando el script correspondiente (desde Windows):

\texttt{./[nombreentorno]/Scripts/activate}

Desde allí, podemos instalar las librerías y dependencias. Se incluye un fichero \textit{requirements.txt} con las librerías en la raíz de ambos directorios (API y APP-WEB), por lo que su instalación es sencilla con el comando pip:

\texttt{pip install -r requirements.txt}

Para salir del entorno y pasar al principal (PATH de nuestro S.O), ejecutaremos \texttt{deactivate}.

\imagen{venv}{Activación y desactivación de entorno virtual}

\section{Compilación, instalación y ejecución del proyecto}

El proyecto está elaborado principalmente en python, lenguaje \textit{interpretado}, por lo que tan sólo se requiere un intérprete de python con las librerías necesarias para su ejecución, no siendo necesaria su \textbf{compilación}.

Por otro lado, se ofrece un \textit{despliegue} en Amazon Web Services desde al que acceder al mismo. El enlace a la aplicación web se encuentra actualizado en la sección \href{https://github.com/jesgararm/GestorQuirofanos#readme}{README} del repositorio:

\imagen{readme}{Localización de enlace al despliegue de la aplicación en GitHub}

Por tanto, en este apartado se describen los pasos a seguir para \textit{desplegar} la aplicación en AWS.

\subsection{Primer paso: Configuración de base de datos}

\subsubsection{Creación e inicio de Base de Datos en AWS}

Accederemos, tras iniciar sesión, a \href{https://aws.amazon.com/es/rds/}{Amazon RDS}, que es el servicio de base de datos relacionales. Una vez allí, crearemos una nueva base de datos (\textit{existe un enlace directo a la funcionalidad desde el panel principal}), seleccionando MySQL y configurando los datos de usuario maestro según el contenido del fichero \texttt{./APP-WEB/src/config.py}, que en nuestro caso se corresponden con \textit{root} como nombre de usuario y \textit{gestorquirofanos} como contraseña.


\imagen{rdscreate}{Configuración de usuario maestro en base de datos RDS}

El resto de los parámetros pueden mantenerse inalterables hasta finalizar el formulario.

\subsubsection{Obtención de parámetros de conexión e importación del esquema}

Tras la creación, desde el panel \textit{Bases de Datos} de la barra de herramientas en RDS, obtendremos un listado de todas las BBDD relacionales creadas.

Al seleccionar la recién creada, acudiremos a un panel que muestra las características de la misma. Desde allí, en el apartado \texttt{Conectividad y Seguridad > Punto de enlace y puerto}, obtendremos los parámetros de conexión para la gestión de la base de datos desde un SGBD.

\imagen{panelrds}{Panel de configuración y características de BD en Amazon RDS}

Tras su obtención, nos conectaremos a la misma, creando una nueva conexión y añadiendo los valores de enlace y puerto, así como el usuario y contraseña.

\imagen{ejemploMySQLWorkbench}{Ejemplo de conexión a BD RDS usando MySQL Workbench}

Tras establecerse la conexión, deberemos crear un \textit{nuevo schema}, denominado \textit{gestor\_quirófanos} y, tras seleccionarlo, ejecutar el script \texttt{./schema\_gestor\_quirofanos.sql}.

Tras la ejecución del script, se crearán las tablas y las dependencias funcionales entre ellas, así como habrá algunos usuarios, predicciones y planificaciones de prueba.

\imagen{schemaGestorQuirofanos}{Vista de script y resultado de su ejecución en conexión a BD RDS}

\subsubsection{Actualización de parámetros de conexión a BD}

Es en el fichero \texttt{./APP-WEB/src/config.py} donde se especifican los parámetros de conexión de nuestro interfaz con la base de datos.

Allí encontramos dos definiciones en la clase \textbf{\textit{DevelopmentConfig}}, una para la conexión con la BD local y otra para la remota.

Para configurarlo, bastará con dejar comentada la porción de código que no nos interesa y actualizar los datos de conexión en la variable \textit{MYSQL\_HOST}.

\imagen{config}{Código fuente con las rutas de conexión con el esquema \textit{gestor\_quirófanos} en la base de datos}

\subsection{Segundo paso: Creación de contenedor API}

Contando con la existencia del \textit{daemon} Docker ejecutándose (basta con iniciar el interfaz Docker Desktop en segundo plano) y los ficheros \textit{Dockerfile}, podremos crear con facilidad desde línea de comandos estos contenedores.

Debemos crear dos contenedores, uno en cada subsistema, a partir del contenido del directorio \texttt{API} y \texttt{APP-WEB}.

Dado que debemos contar con la \textbf{información de conexión de la API} de cara a ejecutar el interfaz web, debemos proceder con estos pasos de forma secuencial para cada uno de ambos subsistemas.

Por ello, nos situamos en \texttt{./API} y ejecutamos \texttt{docker build -t [nombreapi] .}

Tras la ejecución \textit{automática} de los pasos detallados en \textit{Dockerfile}, se creará una \textit{imagen} del contenedor, que podremos ejecutar o desplegar en un servidor.

\imagen{dockerAPI}{Creación de contenedor para la API desde línea de comandos}

\subsection{Tercer paso: Desplegar la API}

\subsubsection{Publicación en Amazon ECR}

Para desplegar la API debemos, en primer lugar, almacenar el docker en un \textbf{repositorio} Amazon, haciendo uso de la herramienta \href{https://aws.amazon.com/es/ecr/}{Elastic Container Registry}.

Debemos crear un nuevo repositorio (\textit{enlace en la ventana principal}), seleccionar la visibilidad pública y añadir un nombre que \textbf{coincida} con el del contenedor, no siendo necesarias más modificaciones.

\imagen{formECR}{Formulario de creación de un repositorio Amazon ECR}

Para el siguiente paso, es conveniente tener instalado el gestor en línea de comandos \href{https://docs.aws.amazon.com/es_es/cli/latest/userguide/getting-started-install.html}{AWS CLI}.

Una vez instalada esta utilidad, seleccionaremos nuestro repositorio y marcaremos la opción \texttt{Mostrar claves de envío}, abriéndose una ventana con los pasos a seguir para publicar la imagen del contenedor en nuestro repositorio:

\imagen{postECR}{Ejemplo de claves de envío Amazon ECR}

Si seguimos los pasos detallados en la ventana, conseguiremos publicar la imagen del contenedor en nuestro repositorio:

\imagen{pushECR}{Ejecución satisfactoria de la subida de un contenedor a repositorio Amazon ECR}

\subsubsection{Despliegue en Amazon ECS}

Una vez almacenado, podremos ejecutar el contenedor en un servidor proporcionado por AWS, mediante la función \textit{Elastic Container Services}.

Deberemos crear (\textit{si no disponemos aún}) un clúster, encargado de ejecutar contenedores a modo de \textbf{servicios y tareas}:

\imagen{vistaClusterTFG}{Visualización de panel de de configuración de nuestro cluster \textbf{ubutfgcluster}}


Crearemos un nuevo \textit{servicio} mediante la opción habilitada para ello en nuestro panel. Podremos dejar todas las opciones marcadas \textbf{por defecto}, debiendo añadir una \textit{definición de tarea} con la URI de la imagen en ECR y el puerto 4000, guardándola con el nombre de \textit{API-TFG}:

\imagen{defTarea}{Definición de familia de tareas API-TFG, añadiendo enlace a la imagen del contenedor en ECR}

Tras la definición de la tarea, volveremos al formulario anterior, donde registraremos el servicio \textit{gestorquirofanosapi}, ligándolo a la familia de tareas API-TFG:

\imagen{defServicio}{Definición de servicio \textbf{\textit{gestorquirofanosapi}} en el cluster}

El servicio arrancará unos instantes después, permitiendo obtener la dirección de acceso a la API en ejecución y pasar al encapsulamiento y despliegue del interfaz.

\subsection{Cuarto Paso: Configuración de rutas de acceso a la API}

Tras arrancar el servicio, deberemos obtener la \textbf{dirección pública} de la tarea en ejecución. Para ello: \texttt{Selección de servicio > Tareas > Tarea > Condiguración > IP pública}

\imagen{tareaIP}{Localización de URL del despliegue de la API}

Una vez allí, en los métodos localizados en \texttt{./APP-WEB/src/public/routes.py} donde se especifique la conexión con la API  (\textit{upload y uploadSched}), comentaremos la línea de conexión con el servidor local y añadiremos:

\begin{itemize}
    \item \textit{http://[url]:4000/predict}: Para las funciones de predicción.
    \item \textit{http://[url]:4000/schedule}: Para las funciones de planificación.
\end{itemize}

\imagen{modRoute}{Modificación de las rutas}

\subsection{Quinto Paso: Encapsulamiento en Docker y despliegue}

De forma similar al paso anterior, nos situamos en línea de comandos en el directorio \texttt{./APP-WEB} y ejecutamos:

\texttt{docker build -t gestorquirofanosapp .}

Tras su ejecución, crearemos un nuevo repositorio, de nombre \textit{gestorquirofanosapp}, siguiendo los comandos del \textbf{tercer paso} (\textit{aunque sustituyendo el nombre gestorquirofanosapi por gestorquirofanosapp}):

\imagen{postECR2}{Comandos de envío del contenedor \textbf{\textit{gestorquirofanosapp}} al nuevo repositorio ECR}

Por último, deberemos definir una nueva tarea (que hemos nombrado como \textit{APP-TFG}) y seleccionarla como base de un nuevo servicio en el clúster, que llamaremos \textit{gestorquirofanosapp}, estableciendo el puerto 5000.

Una vez iniciado el servicio, accederemos a la tarea y a su dirección, siendo ésta la base para la construcción de la \textbf{URL del despliegue de nuestra aplicación web}:

\textit{URL = http://[url-tarea]:5000}

\imagen{urlAPP}{Obtención de la dirección pública del despligue de la APP}

\subsection{Sexto Paso: Comprobación del despliegue}

Accedemos a la URL y comprobamos su funcionamiento:

\imagen{despliegueAPP}{Comprobación del correcto despliegue del sistema en AWS}

\section{Pruebas del sistema}


