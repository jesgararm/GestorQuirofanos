\apendice{Especificación de diseño}

\section{Introducción}

En este apartado desglosaremos las estrategias de diseño tomadas en consideración: \textit{conjuntos de datos, clases, procedimientos..}, para el cumplimiento de los requerimientos funcionales y no funcionales reseñados en el primer apartado.

\section{Diseño de datos}

\subsection{Entidades}

\begin{itemize}
    \item \textbf{Usuario (\textit{User})}: Consta de un identificador auto-incrementado como clave primaria, así como atributos para el \textit{nombre}, su \textit{correo electrónico}, contraseña \textit{encriptada} y fechas de \textit{creación} y \textit{modificación}.
    \item \textbf{Planificación:} Consta de un identificador como clave primaria, el identificador del usuario que la creó (\textit{FK}) y la fecha de creación.
    \item \textbf{Predicción:} Sigue la misma estructura de atributos que la entidad \textit{Planificación}.
\end{itemize}

\newpage
\subsection{Diagrama Relacional}

\imagen{diagRel}{Diagrama Relacional}


\subsection{Diagrama E-R}

\imagen{diagERD}{Diagrama Relacional}

\newpage

\section{Diseño procedimental}

En este apartado, se desgranan los procedimientos que permiten especificar el funcionamiento interno de las aplicaciones.

Existen muchos tipos de diagramas que pueden representar estas funcionalidades, aunque para esta ocasión hemos elegido los \textit{diagramas de interacción}  \cite{Britton2005IdentifyingDiagrams}. En él, registraremos el comportamiento de nuestro sistema mediante una secuencia de eventos \textbf{ordenados por tiempo}.

En primer lugar, reflejaremos la acción de \textit{modificar el perfil} de un usuario en nuestro interfaz:

\imagen{UsuarioEst}{Diag Int-Seq: Modificar Perfil de Usuario}

Mostramos ahora las acciones que puede realizar el usuario \textbf{administrador} sobre las acciones de creación, modificación y eliminación de usuario:

\imagen{AdminUs}{Diag Int-Seq: Gestión de Usuarios}

Dentro de este proyecto, podemos diferenciar dos \textbf{funcionalidades:} Planificación de Intervenciones y Predicción de duración, tanto de forma \textit{independiente} como \textit{interrelacionada} (llamada de una funcionalidad a otra en caso de ser requerida).

El usuario puede \textit{crear, visualizar o eliminar} una determinada \textbf{predicción}, siguiendo el siguiente esquema (partiendo del supuesto de un inicio de sesión exitoso):

\imagen{PredUs}{Diag Int-Seq: Gestión de Predicciones}

De modo similar, gestionaremos las \textbf{planificaciones:}


\imagen{PlanUs}{Diag Int-Seq: Gestión de Planificaciones}


\section{Diseño arquitectónico}


