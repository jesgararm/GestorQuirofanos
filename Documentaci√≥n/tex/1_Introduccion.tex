\capitulo{1}{Introducción}

Durante muchos años, la comunidad científica ha tratado de elaborar modelos predictivos que ayuden en tareas de \textbf{organización industrial.}

Los hospitales producen \textbf{servicios}, y desde hace tiempo han empezado a tomar decisiones estratégicas en los servicios que proveen, debido en gran parte al incremento en la oferta de productos sanitarios y nuevos tratamientos disponibles, así como al entorno competitivo que nos rodea.
Por este motivo, las clínicas, como cualquier otra empresa, buscan medidas para reducir costes e impulsar sus demandas financieras \cite{Maleki2023AMoment}.

Si hablamos de \textbf{sanidad privada}, las dos terceras partes de los ingresos de un hospital dependen de las intervenciones quirúrgicas, las cuales constituyen aproximadamente un 40\% del gasto hospitalario \cite{Pham2008SurgicalProblem}.

Los quirófanos forman parte de un entorno muy complejo, conformado por \textbf{múltiples capas}, y condicionado por un gran número de \textit{interacciones sociales, impredictibilidad y baja tolerancia a fallos} \cite{Rothstein2018OperatingEfficiency}. 

Por otro lado, las irregularidades en el flujo de trabajo dentro del área quirúrgica resultan en perjuicios sobre la \textbf{salud mental de los profesionales}, a menudo motivados por factores como: \textit{amenaza de demandas, alta presión para realizar tareas difíciles, conflicto de intereses...}\cite{Wheelock2015TheTeamwork}.

Por este motivo, incrementar la productividad en los quirófanos tendría una gran influencia en el rendimiento financiero de los mismos, sin dejar de lado algunas consideraciones éticas (\textit{reducción de listas de espera, priorización, equidad en reparto de recursos...}), incrementando la calidad de los servicios y la satisfacción de los pacientes de forma directamente proporcional a las mejoras conseguidas.

El gasto asociado a los quirófanos es uno de los más elevados de cada hospital, por lo que la optimización del proceso resulta crucial de cara a una gestión eficiente de los recursos.
No obstante, el diseño de una planificación quirúrgica resulta muy difícil, debido a una enorme incertidumbre respecto a la duración de las cirugías \cite{Celik2023APrinciple}.

Dentro de los problemas de optimización, si analizamos la literatura encontramos que la planificación quirúrgica ocupa un área especial dentro de esta disciplina.
En función del criterio a mejorar, los investigadores han propuesto diferentes soluciones una vez identificados y analizados los problemas \cite{Gur2018ApplicationOverview}.

Podemos contribuir a la investigación desde diversos puntos de vista, adoptando múltiples estrategias.
Por ello, derivado de disciplinas como la minería o el análisis de datos, se presentan estrategias novedosas que permitan contar con iniciativas prometedoras de cara a lograr la \textbf{eficiencia }\cite{Schouten2023OperatingReview}.

Nuestro proyecto tratará de aunar \textbf{ambos conflictos} en el área de gestión de espacios quirúrgicos mediante un proceso de \textit{análisis exploratorio} y posterior \textit{implementación} de un producto software capaz de \textbf{predecir la duración estimada} en base a un conjunto de características aplicables a los pacientes y, en base a esa información, \textit{\textbf{optimizar}} el reparto de los actos quirúrgicos en función de los recursos disponibles y la prioridad asignada.

\newpage



