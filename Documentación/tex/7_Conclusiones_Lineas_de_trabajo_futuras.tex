\capitulo{7}{Conclusiones y Líneas de trabajo futuras}

En este trabajo hemos tratado de ofrecer una herramienta a los gestores sanitarios para mejorar su desempeño en la planificación de pacientes en listas de espera quirúrgicas.

Para ello, hemos modelado un sistema de \textbf{predicción} y otro de \textbf{optimización}, basados en Inteligencia Computacional, así como integrado conocimientos de diversas ramas de la Ingeniería, adquiridos a lo largo de la titulación a través de las distintas asignaturas, además de recopilar nueva información específica y necesaria para la ejecución del proyecto.

Tras revisar los \textit{resultados} y compararlos con otros estudios similares en la bibliografía, podemos concluir que:

\begin{itemize}
    \item Los modelos de aprendizaje basados en \textit{árboles} parecen ser los más adecuados para abordar el problema de predecir la duración de una determinada intervención quirúrgica, en base a las variables consideradas.
    \item Las medidas de error obtenidas \textbf{disminuyen} al incrementar el número de variables, aunque los valores devueltos constituyen un modelo \textit{válido} y en \textit{consonancia} con lo esperable tras revisar la literatura.
    \item La aplicación de un algoritmo genético ofrece resultados \textbf{muy buenos} y en \textbf{tiempo asumible}, comportándose mucho mejor tras la combinación con heurísticas contrastadas y empleando \textit{restricciones blandas} que faciliten la evolución estocástica hacia la convergencia.
    \item La API desarrollada cumple con los objetivos planteados al principio del proyecto y los mecanismos de \textit{integración} permiten su uso y distribución de forma fácil y asumible para los clientes.
\end{itemize}

Por otra parte, existen gran número de factores que no han sido explotados en el desarrollo del proyecto y sirven como piedra angular para posibles implementaciones futuras:

\begin{itemize}
    \item Integración continua con SGBD hospitalarias y consultas automatizadas para obtener las variables de interés.
    \item Ampliación del \textit{dataset} de entrenamiento, incluyendo factores subjetivos que pudiesen influir en la predicción (\textit{código del facultativo}, \textit{duración prevista}...).
    \item Desarrollo de un modelo de predicción, ajustado a cada especialidad, para predecir los intervalos \textit{vacíos} entre cada caso quirúrgico (\textit{limpieza}, \textit{preparación}...).
    \item Integración en sistemas de gestión hospitalaria de RRHH para disponer del personal necesario en cada sala de operaciones que garantice el rendimiento óptimo propuesto por el algoritmo.
    \item Optimización paralela y secuencial, considerando el \textit{circuito completo} de estancia hospitalaria, los recursos disponibles y los costes asociados.
\end{itemize}

Además, una vez desarrollada este \textit{doble servicio}, se ha implementado un ejemplo de interfaz, a modo de \textit{aplicación web}, siguiendo así la tendencia de numerosos sistemas informáticos de los distintos servicios sanitarios de este país, permitiendo su ejecución en un servidor \textbf{interno}, de acceso \textbf{restringido y distribuido} e \textbf{integrable} con el resto de herramientas disponibles en los servicios informáticos de cada región.

En definitiva, hemos desarrollado una \textbf{utilidad} que cumple con los objetivos propuestos al inicio del trabajo, siendo capaz de ofrecer un amplio abanico de posibilidades sobre las que trabajar hasta obtener un sistema completo, robusto y fiable que pueda ser introducido en el futuro dentro de nuestro Sistema Sanitario.