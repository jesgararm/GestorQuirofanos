\apendice{Especificación de diseño}

\section{Introducción}

En este apartado desglosaremos las estrategias de diseño tomadas en consideración: \textit{conjuntos de datos, clases, procedimientos..}, para el cumplimiento de los requerimientos funcionales y no funcionales reseñados en el primer apartado.

\section{Diseño de datos}

\subsection{Entidades}

\begin{itemize}
    \item \textbf{Usuario (\textit{User})}: Consta de un identificador auto-incrementado como clave primaria, así como atributos para el \textit{nombre}, su \textit{correo electrónico}, contraseña \textit{encriptada} y fechas de \textit{creación} y \textit{modificación}.
    \item \textbf{Planificación:} Consta de un identificador como clave primaria, el identificador del usuario que la creó (\textit{FK}) y la fecha de creación.
    \item \textbf{Predicción:} Sigue la misma estructura de atributos que la entidad \textit{Planificación}.
\end{itemize}

\newpage
\subsection{Diagrama Relacional}

\imagen{diagRel}{Diagrama Relacional}


\subsection{Diagrama E-R}

\imagen{diagERD}{Diagrama Relacional}

\newpage

\section{Diseño procedimental}

\section{Diseño arquitectónico}


