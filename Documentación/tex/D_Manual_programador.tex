\apendice{Documentación técnica de programación}

\section{Introducción}

Para realizar un correcto \textbf{análisis} del desarrollo de este proyecto, son necesarias algunas consideraciones respecto al entorno de desarrollo, las dependencias funcionales y las opciones de integración y despliegue.

\section{Estructura de directorios}

Dentro del repositorio, podemos encontrar:
\begin{itemize}
    \item \textit{./}: Directorio raíz. Contiene la licencia, el logo, el fichero \textit{léeme} para la página principal de GitHub, la estructura de base datos para su importación y todos los paquetes y directorios que conforman el proyecto.
    \item \textit{./API}:  Sistema que contiene la API con los servicios de Predicción y Planificación quirúrgica.
    \item ./\textit{API}/\textit{uploads}: Directorio de almacenamiento temporal que recoge los archivos enviados por los usuarios antes de ser procesados y eliminados.
    \item \textit{./API/src}: Código fuente de la API.
    \item \textit{./API/src/scheduling}: Contiene las clases y paquetes encargados de la tarea de planificación, así como el archivo Dockerfile para ser encapsulado.
    \item \textit{./API/src/scheduling/Optimizacion}: Directorio con clases y paquetes con los algoritmos de optimización y planificación.
    \item \textit{./API/src/scheduling/Optimizacion/Genético}: Clases para el algoritmo genético.
    \item \textit{./API/src/scheduling/Optimizacion/Heurísticas}: Clases para las heurísticas de planificación.
    \item \textit{./API/src/predictions}: Directorio con clases encargadas de cargar el modelo de ML predictivo.
    \item \textit{./API/src/common}: Clases comunes a todos los paquetes. Encargados fundamentalmente de tareas de procesamiento de datos y ficheros.
    \item \textit{./APP-WEB}: Contiene el Dockerfile y el código fuente de la interfaz web.
    \item \textit{./APP-WEB/src}: Código fuente de la aplicación.
    \item \textit{./APP-WEB/src/admin}: Contiene las rutas a las funcionalidades del usuario administrador.
    \item \textit{./APP-WEB/src/auth}: Contiene las rutas para la funcionalidad de autenticación en sistema.
    \item \textit{./APP-WEB/src/forms}: Contiene los formularios en formato Flask WTF para incluirlos en las plantillas HTML.
    \item \textit{./APP-WEB/src/models}: Clases y paquetes que encapsulan el modelo de comunicación con la base de datos.
    \item \textit{./APP-WEB/src/models/entities}: Contiene las clases que representan las entidades persistentes de la BBDD.
    \item \textit{./APP-WEB/src/public}: Contiene las rutas para todas las funcionalidades disponibles para los usuarios.
    \item \textit{./APP-WEB/src/static}: Contiene los archivos de formato de estilos, imágenes... a incluir en las plantillas estáticas.
    \item \textit{./APP-WEB/src/templates}: Contiene las plantillas en lenguaje de marcado HTML, que se corresponden con la \textit{Vista} del usuario.
    \item \textit{./APP-WEB/src/templates/admin}: Plantillas dedicadas a la vista del administrador.
    \item \textit{./APP-WEB/src/templates/user}: Plantillas para la vista de cualquier usuario.
    \item \textit{./APP-WEB/src/templates/auth}: Plantillas para la función de identificación en el sistema.
    \item \textit{./Documentación}: Se incluye la memoria, anexos y diagramas.
    \item \textit{./Documentación/img}: Ruta de las imágenes que se encuentran en el entregable.
    \item \textit{./Documentación/tex}: Capítulos y apartados de memoria y anexos, en formato latex.
    \item \textit{./Documentación/UML}: Directorio con diagramas de casos de uso, interacción y secuencia.
    \item \textit{./Experimentación}: Colección de Jupyter Notebooks que ilustran el proceso de investigación llevado a cabo hasta obtener la solución propuesta.
    \item \textit{./Experimentación/Datos}: Colección de conjuntos de datos anonimizados para labores de ML.
    \item \textit{./Experimentación/Modelos}: Análisis, diseño y explotación de modelos predictivos de aprendizaje supervisado.
    \item \textit{./Experimentación/Optimización}: Análisis, diseño y explotación de algoritmos de planificación paralela.
    \item \textit{./Experimentación/Preprocesado:} Herramientas de preprocesamiento de datos a partir de los listados originales.
    \item \textit{./Preprocesado}: Clase con utilidades para estandarizar y homogeneizar las fuentes de datos en un formato compatible con los modelos propuestos.
\end{itemize}


\section{Manual del programador}



\section{Compilación, instalación y ejecución del proyecto}

\section{Pruebas del sistema}


