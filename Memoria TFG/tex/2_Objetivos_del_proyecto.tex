\capitulo{2}{Objetivos del proyecto}

El objetivo \textbf{principal} del proyecto consiste en la elaboración de un \textit{sistema de gestión de quirófanos} que permita al usuario la introducción de un set de pacientes listos para intervenir, junto a una planificación de los recursos disponibles, y genere \textbf{automáticamente} una propuesta de planificación en base a los tiempos estimados y los recursos disponibles.

\subsection{Objetivos del Proyecto de Software}
Podemos introducir algunos de los objetivos principales con los que debe contar nuestra herramienta.
\tablaSmall{Objetivos del Proyecto de Software}{c}{ObjetivosSoftware}
{\textbf{Listado de Objetivos Software}\\}{ 
 Predecir el tiempo estimado de un acto quirúrgico. \\ 
 Planificar quirófanos en función del tiempo predicho y los recursos disponibles. \\ 
 Recibir datos relativos a pacientes. \\
 Validar los datos.\\
 Ofrecer una salida al cliente. \\ 
 } 

Específicamente, podemos subdividir los objetivos en función de una de las tres áreas principales a implementar, tal y como podemos consultar en la siguiente figura:

\imagen{Memoria TFG/img/objetivosProyecto.png}{Subdivisión del Proyecto}{0.8}

\newpage

\section{Objetivos para la Predicción de la Duración}

Hacen referencia a la labor de Machine Learning para la predicción de la duración de una intervención quirúrgica en base a los datos introducidos por el usuario de un set de pacientes y la labor de aprendizaje previa efectuada por nuestro modelo.

Podemos consultar los objetivos en la tabla adjunta.

\tablaSmall{Objetivos en Predicción de la Duración}{c}{ObjetivosDuracion}
{\textbf{Listado de Objetivos Optimización}\\}{ 
Obtener un dataset de pacientes intervenidos en un hospital. \\ 
 Preprocesar los datos de entrada y obtener variables. \\ 
 Analizar la distribución de los datos. \\
 Modelar como variable objetivo como regresión y/o clasificación.\\
 Identificar modelos predictivos que se ajusten a la entrada. \\ 
 Calcular medidas de error. \\
 Evaluar los resultados. \\
 Elegir y exportar el modelo en base al error. \\
 } 

\section{Objetivos para la Optimización en la Planificación}

Son los objetivos marcados para la elaboración de un algoritmo de optimización que gestione adecuadamente los recursos hospitalarios y asigne eficientemente los huecos de quirófano a los pacientes según su \textbf{prioridad} y las métricas de \textbf{rendimiento} elegidas.

\tablaSmall{Objetivos en Optimización de la Planificación}{c}{ObjetivosPlanificacion}
{\textbf{Listado de Objetivos Predicción}\\}{ 
 Recoger un set de pacientes/intervenciones, su duración y prioridad. \\ 
 Recoger un listado de salas de operaciones y tiempo disponibles. \\ 
 Identificar métricas de rendimiento a evaluar. \\
 Modelar el problema de programación lineal.\\
 Identificar metaheurísticas que resuelvan el problema modelado. \\ 
 Remodelar el problema en base a la metaheurística y evaluar. \\
 Analizar resultados y seleccionar la metaheurística. \\
 Exportar el modelo de optimización para su uso en la interfaz. \\
 } 

\section{ Objetivos de la API}

Son aquellos que debe cumplir nuestra integración para satisfacer las especificaciones marcadas en el proyecto.

\tablaSmall{Objetivos de la Interfaz de Usuario}{c}{ObjetivosInterfaz}
{\textbf{Listado de Objetivos API}\\}{ 
 Introducir listado de pacientes a partir de un fichero externo. \\
 Cargar modelo predictivo y calcular tiempos estimados. \\
 Introducir número de quirófanos, ventanas temporales y días a planificar. \\
 Cargar modelo de optimización y planificar según los datos.\\
 Devolver planificación al usuario.\\
 Devolver predicción temporal al usuario. \\
 } 
