\capitulo{1}{Introducción}

Durante muchos años, la comunidad científica ha tratado de elaborar modelos predictivos que ayuden en tareas de \textbf{organización industrial.}

La necesidad de precisar métodos de planificación y producción que sean eficientes en tiempo y coste son cada día más importantes en los \textbf{mercados} competitivos actuales \cite{Geurtsen2023ProductionReview}. Para los fabricantes, resulta especialmente crucial disponer de métodos que garanticen la \textbf{optimización} del equipamiento y los recursos empleados, aunque siempre considerando que, en condiciones reales, pueden sucederse eventos impredecibles que hagan peligrar la viabilidad del modelo \cite{Geurtsen2023ProductionReview}.

El gasto asociado a los quirófanos es uno de los más elevados de cada hospital, por lo que la optimización del proceso resulta crucial de cara a una gestión eficiente de los recursos.
No obstante, el diseño de una planificación quirúrgica resulta muy difícil, debido a una enorme incertidumbre respecto a la duración de las cirugías \cite{Celik2023APrinciple}.

