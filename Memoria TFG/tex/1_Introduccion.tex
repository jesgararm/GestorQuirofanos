\capitulo{1}{Introducción}
Durante muchos años, la comunidad científica ha tratado de elaborar modelos predictivos que ayuden en tareas de \textbf{organización industrial.}

La necesidad de precisar métodos de planificación y producción que sean eficientes en tiempo y coste son cada día más importantes en los \textbf{mercados} competitivos actuales \cite{Geurtsen2023ProductionReview}. Para los fabricantes, resulta especialmente crucial disponer de métodos que garanticen la \textbf{optimización} del equipamiento y los recursos empleados, aunque siempre considerando que, en condiciones reales, pueden sucederse eventos impredecibles que hagan peligrar la viabilidad del modelo \cite{Geurtsen2023ProductionReview}.

Los hospitales producen \textbf{servicios}, y desde hace tiempo han empezado a tomar decisiones estratégicas en los servicios que proveen, debido en gran parte al incremento en la oferta de productos sanitarios y nuevos tratamientos disponibles, así como al entorno competitivo que nos rodea.
Por este motivo, las clínicas, como cualquier otra empresa, buscan medidas para reducir costes e impulsar sus demandas financieras \cite{Maleki2023AMoment}.

Si hablamos de \textbf{sanidad privada}, las dos terceras partes de los ingresos de un hospital dependen de las intervenciones quirúrgicas, las cuales constituyen aproximadamente un 40\% del gasto hospitalario\cite{Pham2008SurgicalProblem}.

Los quirófanos forman parte de un entorno muy complejo, conformado por \textbf{múltiples capas}, y condicionado por un gran número de \textit{interacciones sociales, impredictibilidad y baja tolerancia a fallos.} \cite{Rothstein2018OperatingEfficiency}. 

Por otro lado, las irregularidades en el flujo de trabajo dentro del área quirúrgica resultan en perjuicios sobre la \textbf{salud mental de los profesionales}, a menudo motivados por factores como: \textit{amenaza de demandas, alta presión para realizar tareas difíciles, conflicto de intereses...}\cite{Wheelock2015TheTeamwork}.

Por este motivo, incrementar la productividad en los quirófanos tendría una gran influencia en el rendimiento financiero de los mismos, sin dejar de lado algunas consideraciones éticas (\textit{reducción de listas de espera, priorización, equidad en reparto de recursos...}), incrementando la calidad de los servicios y la satisfacción de los pacientes de forma directamente proporcional a las mejoras conseguidas.

El gasto asociado a los quirófanos es uno de los más elevados de cada hospital, por lo que la optimización del proceso resulta crucial de cara a una gestión eficiente de los recursos.
No obstante, el diseño de una planificación quirúrgica resulta muy difícil, debido a una enorme incertidumbre respecto a la duración de las cirugías \cite{Celik2023APrinciple}.

Dentro de los problemas de optimización, si analizamos la literatura encontramos que la planificación quirúrgica ocupa un área especial dentro de esta disciplina.
En función del criterio a mejorar, los investigadores han propuesto diferentes soluciones una vez identificados y analizados los problemas \cite{Gur2018ApplicationOverview}.

Podemos contribuir a la investigación desde diversos puntos de vista, adoptando múltiples estrategias.
Por ello, derivado de disciplinas como la minería o el análisis de datos, se presentan estrategias novedosas que permitan contar con iniciativas prometedoras de cara a lograr la \textbf{eficiencia }\cite{Schouten2023OperatingReview}.
\newpage

\subsection{Métricas de Rendimiento}
Antes de desarrollar cualquier medida que tenga como objetivo alcanzar una mejora de \textbf{rendimiento en la planificación quirúrgica (\textit{OR performance})}, debemos definir qué entendemos por rendimiento y, en su caso, \textit{cómo medirlo}. 

Algunas de las más importantes son los referidos en \ref{Métricas de Quirófano}.
\begin{table}[]
    \centering
    \begin{tabular}{c|c}
        \toprule
            \textbf{Métrica}   &  \textbf{Descripción}  \\
         \midrule
              \textit{Hora de Comienzo}  &  Retraso o anticipo en el comienzo de la actividad quirúrgica. \\
              \textit{Tiempo de Recambio }& Tiempo entre cada uno de los pacientes. \\
              \textit{Uso del bloque} & Proporción del tiempo asignado que se dedica a la realización de la actividad. \\
              \textit{Tasa de Cancelaciones} & Las suspensiones o cancelaciones suponen tiempo de actividad desaprovechado. \\
       \bottomrule
    \end{tabular}
    \caption{Métricas de Quirófano}
    \label{Métricas de Quirófano}
\end{table}
No obstante, las métricas usadas para cuantificar el rendimiento de las medidas de optimización en quirófano son muy diversas a lo largo de la literatura.
Muchos artículos focalizan la eficiencia en aspectos meramente cualitativos, mientras que otros se centran en mejorar las métricas más objetivas, buscando la minimización del malgasto de recursos y la correcta gestión del tiempo. \cite{Schouten2023OperatingReview}.

Podemos resumir, por tanto, que un correcto rendimiento debe satisfacer requisitos de \textbf{eficiencia }y \textbf{calidad} \cite{Sandbaek2014ImpactEfficiency} , referidas en \ref{Eficiencia y Calidad}
\begin{table}[]
    \centering
    \begin{tabular}{c|c}
       \textbf{Eficiencia}  &  \textbf{Calidad} \\
        Maximizar utilización y reducción de retrasos y esperas. & Calidad en cuidados, Seguridad del paciente y Bienhacer profesional. 
    \end{tabular}
    \caption{Eficiencia y Calidad}
    \label{Eficiencia y Calidad}
\end{table}
\newpage
\subsection{Predicción de Duración}

