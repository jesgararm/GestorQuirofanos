\capitulo{5}{Aspectos relevantes del desarrollo del proyecto}

\section{Obtención y Preprocesamiento de Datos}

\subsection{Fuente de Datos}

Se obtuvieron datos relativos a intervenciones quirúrgicas realizadas en el \textbf{Hospital Universitario Virgen del Rocío} de Sevilla, comprendidas entre las fechas \textit{1 de Enero de 2016} y \textit{1 de Noviembre de 2022}.

Los datos fueron recogidos desde la Intranet Hospitalaria, al pertenecer el autor de este proyecto al personal sanitario del mismo (\textit{Factultativo Especialista de Área UGC Cirugía Plástica y Grandes Quemados}) y contar con \textbf{autorización} para el uso de datos clínicos con fines académicos.

Se obtuvieron datos relativos a las siguientes \textbf{especialidades quirúrgicas}:
\begin{enumerate}
    \item Cirugía Plástica y Reparadora.
    \item Cirugía Oral y Maxilofacial
    \item Neurocirugía.
    \item Cirugía General y del Aparato Digestivo.
    \item Cirugía Ortopédica y Traumatología.
    \item Otorrinolaringología.
    \item Cirugía Torácica
\end{enumerate}

En total, se consideraron \textbf{25492 individuos} tras la realización de labores de adecuación e integración de las fuentes de datos.

\tablaSmall{Tamaño Muestral por Especialidades}{c|c}{n_espec}
{ \textbf{Especialidad Quirúrgica}  &  \textbf{N} \\}{ 
 Cirugía Plástica y Reparadora & 4761\\ 
Cirugía Oral y Maxilofacial & 3206 \\
 Neurocirugía & 1934\\
 Cirugía General y del Aparato Digestivo & 3263\\
 Cirugía Ortopédica y Traumatología & 9390\\
 Otorrinolaringología & 2142\\
 Cirugía Torácica & 797\\}

 \subsection{Limpieza y Preparación de los Datos}

 Tras obtener los \textbf{listados} de la Base de Datos anteriormente mencionada, se realizó una labor de \textit{adecuación} de los mismos a nuestro entorno de trabajo.

 Con una sucesión de \textit{herramientas} de edición de \textit{DataFrames}\footnote{Un Dataframe es una estructura de datos bidimensional que permite destacar relaciones entre las distintas variables de una Serie de datos. }, proporcionadas por la librería Pandas, se realizó el proceso de \textbf{limpieza de los datos: }
 \begin{enumerate}
     \item Eliminación de caracteres innecesarios.
     \item Extracción de variables.
     \item Eliminación de datos personales.
 \end{enumerate}

 \imagen{previaPreprocesado}{Aspecto de la Base de Datos antes de la edición.}{.9}

\imagen{postLimpieza}{Aspecto de los datos tras la limpieza y edición}{.9}

\subsection{Preprocesado}
 