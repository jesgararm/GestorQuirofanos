\capitulo{5}{Aspectos relevantes del desarrollo del proyecto}

\section{Obtención y Preprocesamiento de Datos}

\subsection{Fuente de Datos}

Se obtuvieron datos relativos a intervenciones quirúrgicas realizadas en el \textbf{Hospital Universitario Virgen del Rocío} de Sevilla, comprendidas entre las fechas \textit{1 de Enero de 2016} y \textit{1 de Noviembre de 2022}, con adecuada anonimización de los datos que garanticen su uso en el ámbito universitario.

Se obtuvieron datos relativos a las siguientes \textbf{especialidades quirúrgicas}:
\begin{enumerate}
    \item Cirugía Plástica y Reparadora.
    \item Cirugía Oral y Maxilofacial
    \item Neurocirugía.
    \item Cirugía General y del Aparato Digestivo.
    \item Cirugía Ortopédica y Traumatología.
    \item Otorrinolaringología.
    \item Cirugía Torácica
\end{enumerate}

\imagen{intervencionesEspecialidad}{Número de Intervenciones Registradas por Especialidad}{.9}

En total, se consideraron \textbf{25492 individuos} tras la realización de labores de adecuación e integración de las fuentes de datos.

 \imagen{tiposDatos}{Variables del Conjunto de Datos Base}{.9}

Por otro lado, contamos con otro dataset, de menor tamaño y limitado a una única especialidad (\textit{Cirugía Plástica y Reparadora}), que \textbf{amplía} el número de variables.
 
  \imagen{datosPlastica}{Variables del Conjunto de Datos Ampliado}{.9}


 \subsection{Preparación y Estadísticos}

 Tras obtener los \textbf{listados} de la Base de Datos anteriormente mencionada, se realizó una labor de \textit{adecuación} de los mismos a nuestro entorno de trabajo, \cite{McKinney2010DataPython}.

 \imagen{previaPreprocesado}{Aspecto de la Base de Datos antes de la edición.}{.9}

\imagen{postLimpieza}{Aspecto de los datos tras la limpieza y edición}{.9}

 Una vez obtuvimos nuestro dataset listo para la realización de labores de \textbf{minería}, procedimos a efectuar un \textit{análisis preliminar}, donde cabe destacar que la \textit{media} de duración de cada intervención \textbf{no varía} de forma estadísticamente significativa en función de la especialidad (\textit{ANOVA-Test}\footnote{Análisis de la varianza. Test estadístico realizado para valorar si existen diferencias significativas en las medias de los valores entre varios grupos}.)

 \imagen{duracionMedia}{Media de duraciones de intervenciones por Especialidad}{.9}

 Por otro lado, efectuamos un estudio de la \textbf{correlación} entre algunas de las variables registradas, centrándonos en los valores devueltos por el \textit{coeficiente de correlación de Spearman}\cite{Page1963OrderedRanks}. 
 Tras esta evaluación, encontramos que los valores referentes al \textbf{tipo, código de intervención y prioridad} son aquellos que, a priori, se encuentran más relacionados con la \textbf{duración}.

\imagen{correlacionSpearman}{Matriz de Correlación de Spearman}{.9}


 \subsection{Preprocesamiento de Datos}

 La mayor parte de los algoritmos de aprendizaje supervisado, especialmente aquellos que emplearemos en nuestro estudio, requieren contar con un conjunto de variables \textbf{numéricas} como conjunto de \textit{características}:
 
 \begin{enumerate}
     \item Para las variables \textbf{binarias}, se realizó una codificación a nivel de bit (0,1).
     \item Para las variables \textbf{ordinales}, se asignó una escala \textit{creciente} en función de la categoría.
     \item Para las variables \textbf{nominales}, usamos la técnica de \textit{One Hot Encoding}\footnote{Técnica que consiste en la división de una variable categórica en cada uno de sus posibles valores, realizando una asignación binaria en función del mismo a la instancia de entrenamiento.}, al tratarse de una de las técnicas más recomendadas para el manejo de variables categóricas en el entrenamiento de modelos de aprendizaje automático\cite{Potdar2017AClassifiers}.
 \end{enumerate}

 El seguimiento y proceso tanto de los procesos de \textbf{codificación} y \textbf{procesado}, así como de exportación posterior, se encuentra detallado en los Jupyter Notebooks: \textit{.\\Datos\\PreprocesadoDatos.ipynb} y \textit{.\\Datos\\codificacionDatos.ipynb}.

Por último, de entre todas las estrategias\cite{Emmanuel2021ALearning} existentes para lidiar con los valores \textbf{ausentes} durante la fase de procesamiento, optamos por la \textbf{eliminación} de instancias, dado que el carácter arbitrario de las labores de codificación y la \textit{escasez} de variables correladas dificultan la aplicación de métodos basados en \textit{sustitución o interpolación}. 

\newpage

\section{Aprendizaje Supervisado: Predicción}

En este apartado se describe el procedimiento de exploración, análisis e implementación de diferentes modelos predictivos, basados en el paradigma de \textit{aprendizaje supervisado} con el objetivo de obtener la predicción más \textbf{fidedigna} posible, de forma que pudiese ser empleada como entrada del modelo de optimización posterior.

\subsection{Dificultades Iniciales}

Tras comenzar la implementación de varios modelos con el \textbf{dataset principal}, observamos que las métricas de error devolcían unos valores \textit{desproporcionados}, en torno a miles de minutos de error medio, muy diferentes  otorgados por otros estudios\cite{ShahabiKargar2014PredictingSurgery}.

Esta situación nos hizo sospechar de la presencia de numerosos \textit{outliers} en nuestro conjunto de datos inicial. 

Este problema fue solucionado haciendo uso del IQR\footnote{Técnica de selección de datos del conjunto de entrenamiento que escoge aquellas instancias con valores de la variable dependiente comprendidos entre 1.5 veces por debajo y por encima de los cuartiles primero y tercero, respectivamente.}\cite{Bonthu2021DetectingOutliers}.

Como podemos comprobar, la ejecución muestra unos valores mucho más coherentes y concordantes de la literatura, constituyendo un buen punto de partida para nuestro ensayo.

\subsection{Explotación de los Modelos}

Los modelos de aprendizaje han sido aplicados siguiendo las bases teóricas especificadas en apartados anteriores y basándonos en la implementación de la librería \textit{scikit-learn}\cite{2021Scikit-LearnPython}, ejecutándose en ambos conjuntos de datos.

Para el ajuste de los hiperparámetros, empleamos un modelo de validación cruzada de tipo \textit{GridSearch}\footnote{Técnica de validación cruzada incluida en scikit-learn, consistente en extraer los mejores valores y combinaciones introducidos en la cuadrícula de parámetros.}.

Comenzamos aplicando un modelo simple de \textbf{regresión múltiple}, donde encontramos diferencias entre ambos conjuntos,  disminuyendo el RMSE un 20\% e incrementando un 50\% el índice $R^{2}$.

No obstante, tras aplicar un \textbf{árbol de regresion},  obtuvimos una \textit{reducción de la raíz del error cuadrático medio} ligeramente superior al 20\%, con escasas diferencias entre ambos datasets.

\imagen{ejemploRegressionTree}{Sección del Árbol de Regresión Construido}{.9}

La ejecución de KNN\footnote{K-Vecinos más cercanos, con K=14 y distancia de Manhattan (\(\sum_{i=1}^{n} |p_{i} - q_{i}|\))} y MLP\footnote{Perceptrón Multicapa} ofrecieron métricas de error son \textit{similares} a las otorgadas por el algoritmo de regresión lineal, con un tiempo de ejecución \textbf{sustancialmente mayor}.

Los resultados ofrecidos por el ensemble de \textbf{random forest} fueron muy similares a los devueltos por el árbol de regresión, tanto en métricas de error como en tiempo de ejecución sobre el conjunto de validación.

Por último, tratamos de convertir el planteamiento del problema en uno de \textit{clasificación}, dividiendo la duración en intervalos, pero las medidas de validez (\textit{precisión, AUC...}) y la escasez de ejemplos en la literatura que empleasen este enfoque nos hicieron centrarnos en el enfoque inicial.


\subsection{Comparación y seleccion del modelo}

Tras el entrenamiento y validación de los algoritmos de aprendizaje, comparamos las métricas calculadas y seleccionamos aquel modelo que mejor se ajusta a los \textbf{objetivos} planteados en el proyecto.

\tablaSmall{Comparativa de algoritmos ML}{c|c|c|c}{compModelos}{
\textbf{\textit{Modelos}} & \textit{MSE} & \textit{RMSE} & \textit{MAE} \\
}{
\textbf{Regresión Lineal} & 3461.43 & 58.83 & 43.27\\
\textbf{Árbol de Regresion} & 2143.10 & 46.29 & 33.48\\
\textbf{KNN} & 3084.32 & 55.54 & 39.82\\
\textbf{Random Forest} & 2364.55 & 48.63 & 34.75 \\
\textbf{Perceptrón Multicapa} & 2804 & 52.96 & 40.46\\
}

En base a estos resultados\footnote{Para más detalles, consultar los Jupyter Notebooks incluidos en la carpeta 'Modelos' del respositorio.}, decidimos seleccionar el \textbf{árbol de regresión}, ya entrenado y guardado\footnote{Las librerías \textit{pickle} y \textit{joblib} permiten serializar y deserializar objetos.} para su uso como servicio por los clientes.


\imagen{decTree}{Representación de rendimiento del árbol de regresión}{.9}

\newpage

\section{Optimización: Planificación y Gestión}


Una vez seleccionado y entrenado el algoritmo de predicción, pasamos a diseñar y evaluar diferentes alternativas para optimizar el sistema de planificación quirúrgica.

Establecimos dos objetivos principales en este apartado:
\begin{enumerate}
    \item \textbf{Maximizar} la prioridad de los casos asignados a las salas de operaciones.
    \item \textbf{Minimizar} la aparición de huecos libres.
\end{enumerate}

Modelamos el problema a través de la siguiente \textit{función de fitness}:
\begin{equation}
    min \sum_{i=1}^{N} \sum_{j=1}^{N}\frac{L_{ij}}{P_{ij}}
\end{equation}
Donde $L$ representa los huecos libres en el quirófano $i$ el día $j$ y $P$ la suma de prioridades de los casos asignados, debiendo cumplirse que:
\begin{equation}
    L_{i}\geq 0
\end{equation}

Tal y como pudimos comprobar en la sección \ref{heur}, se adaptaron en primer lugar dos heurísticas\cite{Lin2020AScheduling}. 
En el enfoque original, se consideraba una fecha límite para priorizar los casos en dos grupos, ordenándolos por tiempo en una cola. Nosotros dividimos los valores de prioridad en deciles y, una vez allí, reordenamos por tiempo antes de incluirlos en la cola.

Posteriormente, codificamos un \textbf{algoritmo genético}, siguiendo una disposición cromosómica \textit{vectorial}, usando separadores para diferenciar quirófanos y días y generando individuos \textbf{válidos} para la población inicial.
A la función de \textit{evaluación} se le añadió una comprobación de \textit{validez} y \textit{distancia} \cite{CoelloCoello2002TheoreticalArt}.

Para la \textit{selección}, empleamos el \textbf{torneo con elitismo}, como mecanismo de \textit{cruce}, adaptamos el concepto de \textbf{cruce ordenado en dos puntos} al diseño de nuestro individuo\cite{Lin2020AScheduling} y para la mutación el \textbf{intercambio en un punto}.

 Cabe destacar que introducimos en la población dos individuos generados a partir de las heurísticas anteriores, con el objetivo de \textbf{acelerar} la convergencia.

 Por último, probamos una implementación del algoritmo \textit{PSO}\footnote{Optimización de enjambre de partículas:  Algoritmo bio-inspirado iterativo, evocando el comportamiento de las partículas en la naturaleza.}, aunque sus valores de validez \textbf{no alcanzaron }las cotas mínimas de desempeño, probablemente por una inadaptación de nuestro modelaje a la metaheurística.

 \tablaSmall{Resultados con N=100, 3 Quirófanos y 5 días}{c|c|c}{compOpti}{ \textbf{Modelos} & \textbf{Fitness} & \textbf{Tiempo}\\}{
 \textit{LPT} & 0,0043 & 0,001s\\
 \textit{LPT/EDD} & 0,0049 & 0,001s\\
 \textit{Genético Random} & 0,00252 & 20.52s\\
 \textit{Genético + Heurísticas} & 0,00203& 17,52s\\
 }

 Como podemos comprobar, el algoritmo que mejor responde a nuestros objetivos, siguiendo un \textbf{tiempo de ejecución razonable} se basa en la combinación de las heurísticas con el enfoque genético, motivo por el cual ha sido el \textbf{seleccionado} para el desarrollo del producto.

 \imagen{fitnessGen}{Evolución de la función de evaluación en el modelo seleccionado}{.9}

 \newpage

 \section{Integración y Despliegue}

 Una vez explorados y seleccionados los algoritmos base para la consecución de los \textbf{objetivos} planteados en nuestro trabajo, decidimos integrar el desarrollo en una API.

 Para su desarrollo, empleamos el framework \textit{Flask}\cite{Grinberg2018FlaskPython} para Python, encapsulando los siguientes \textbf{servicios} para los clientes:
 
 \begin{enumerate}
     \item Recepción y validación de listados en formato CSV o JSON.
     \item Predicción de duración prevista de intervenciones.
     \item Planificación quirúrgica en base al número de quirófanos y días introducidos por el usuario.
     \item Entrega de resultados en formato JSON.
 \end{enumerate}

Uno de los principales problemas más frecuentes para el despliegue de herramientas de \textit{Machine Learning} o cualquier otro tipo de solución computacional consiste en adaptar los equipos/servidores de los clientes al entorno empleado para su desarrollo (\textit{lenguajes, librerías, OS...}). 

Como solución a este paradigma, una vez desarrollada y lista para su explotación, planteamos para su \textbf{despliegue} recurrir a un \textit{Docker}\cite{Merkel2014Docker:Deployment} que contenga tanto los recursos del sistema operativo como los lenguajes y librerías necesarios para poder proveer del servicio desarrollado, bien de forma local o remota, garantizando su adecuada \textbf{portabilidad} y \textbf{usabilidad}.

