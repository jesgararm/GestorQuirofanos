\capitulo{4}{Técnicas y herramientas}

\section{Metodología de Desarrollo}

Como metodología de desarrollo de nuestro proyecto, hemos tomado un enfoque \textbf{ágil}, inspirado en Scrum \cite{Palacio2022ScrumMaster}.

Para ello, se han dividido las diferentes tareas en Sprints, coincidiendo la finalización de cada Sprint con una \textbf{reunión} de los tutores y el desarrollador, finalizando en la confección de un \textbf{entregable} (\textit{release}).

\imagen{como-funciona-scrum}{Funcionamiento de Scrum. Fuente:\cite{SaezHurtado2021ComoUtilizarla}}{.8}

\subsection{Plataformas de Apoyo al Desarrollo}

\subsubsection{Repositorio y Git}

El código fuente, así como la \textbf{documentación} (incluyendo la Memoria y los Anexos) se encuentran en un \textit{repositorio Github} público, accesible desde: \href{https://github.com/jesgararm/GestorQuirofanos}{Repositorio: GestorQuirófanos}

Los \textbf{repositorios} constituyen el núcleo del trabajo en GitHub. A menudo, pueden ser representados como un microservicio, documentación, aplicación.. o todo en uno. 

La \textit{visibilidad} del mismo puede ser especificada, pudiéndose añadir \textit{colaboradores} con la posibilidad de editar la información contenida en él.

El \textbf{propietario} del repositorio es el alumno autor del proyecto, nombrándose como colaboradores los tutores del mismo.


\imagen{github}{Características y Posibilidades de GitHub. Fuente: \cite{Jones2018GettingGuide} }{.9}

\subsubsection{Calidad del Código}

Si bien es cierto que el objetivo principal de nuestro proyecto se centra en el diseño y explotación de algoritmos de \textit{inteligencia computacional}, hemos tratado de aplicar \textbf{buenas prácticas} de codificación y mantenimiento del código fuente, apoyándonos en la herramienta \href{https://codeclimate.com/github/jesgararm/GestorQuirofanos}{Code Climate}, debido a la \textit{sencillez} de su interfaz y la \textit{gratuidad} de sus servicios para proyectos \textit{Open-Source} como el nuestro.

Por otra parte, siguiendo el catálogo de Fowler\cite{Fowler1999Refactoring:Code}, se fueron aplicando diferentes \textbf{refactorizaciones} con el objetivo de mejorar las métricas de calidad del código fuente de cara a su integración final como API.


\section{Elementos de Programación}

\subsection{Lenguajes Utilizados}

Como lenguaje de \textbf{programación}, hemos usado Python\cite{VanRossum2009PythonManual}. Python es considerado por muchos como 





